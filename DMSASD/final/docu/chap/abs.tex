\begin{cabstract}
在近年大数据、人工智能等计算机学科的蓬勃发展下,其机器学习领域中的深度学习已经大范围的成功应用于许多从大数据分析到计算机视觉等各种复杂问题。同样的深度学习等演算法的新兴领域也有可能会被使用在造成隐私、民主和严重的国家安全造成威胁的用途上。近期出现的基于深度学习影响最大的应用之一是 "Deepfake",而所谓的 Deepfake 演算法可以创造出人类用肉眼也不容易辨别出真假的影像与照片。因此,在面对个困境来说能够进行自动检测和评估影像、图片、语音等媒体完整性的技术的研究与讨论是必不可少的过程。本作业先说明介绍了人工智能、深度学习与用于深度伪造与检测的背景,第二章再来说明其研究跟工具的分类、第三章则说明目前当下可用的资料集与素材、第四章则将该领域的研究进行归纳整理、第五章则对近来的研究进行说明,最后将这些调查工作进行总结。

该作业所进行得调研工作於此 GitHub 项目 :

https://github.com/kancheng/kan-cs-report-in-2022/tree/main/DMSASD/final

\end{cabstract}

%\begin{eabstract}
%    英文摘要部分...
%\end{eabstract}

% vim:ts=4:sw=4
