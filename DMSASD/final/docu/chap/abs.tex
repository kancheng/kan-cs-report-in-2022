\begin{cabstract}
在近年大數據、人工智能等計算機學科的蓬勃發展下,其機器學習領域中的深度學習已經大範圍的成功應用於許多從大數據分析到計算機視覺等各種複雜問題。同樣的深度學習等演算法的新興領域也有可能會被使用在造成隱私、民主和嚴重的國家安全造成威脅的用途上。近期出現的基於深度學習影響最大的應用之一是 "Deepfake",而所謂的 Deepfake 演算法可以創造出人類用肉眼也不容易辨別出真假的影像與照片。因此,在面對個困境來說能夠進行自動檢測和評估影像、圖片、語音等媒體完整性的技術的研究與討論是必不可少的過程。本作業先說明介紹了人工智能、深度學習與用於深度偽造與檢測的背景,第二章再來說明其研究跟工具的分類、第三章則說明目前當下可用的資料集與素材、第四章則將該領域的研究進行歸納整理、第五章則對近來的研究進行說明,最後將這些調查工作進行總結。

\end{cabstract}

%\begin{eabstract}
%    英文摘要部分...
%\end{eabstract}

% vim:ts=4:sw=4
