\chapter{深度伪造的资料集}
\label{chap:2}

由于深度伪造的技术蓬勃发展,同时为了因应深度伪造技术的往负面应用的过程中,与其类似于攻守关系而逐渐发展起来的深度伪造检测手段的需求,间接带动了各个机沟与研究者对深度伪造与检测领域上的数据集的迫切需要,因此根据 Li XR 等人的工作总结中的所描述的开源数据集的各概况资讯等成果,本作业则将其整理而后条列,同时分为影像资料集与语音资料集如下 :

\section{深度伪造的影像资料集}

\begin{itemize}
\item [-] UADFV \cite{matern2019exploiting}:早期的研究数据来源,使用 FakeAPP \cite{list1141} 工具进行合成,同时在 Youtube 平台搜集素材,其资料分别有 49 真实未修改的影像与 49 个已经修改过的伪造影像。

\item [-] FaceForensics(FF) \cite{rossler2018faceforensics}: 从 Youtube8M \cite{abu2016youtube} 的来源中将与人类脸部有关联的目标中取出 1004 的影像,并用 Face2Face 进行改造 1004 个资料集。

\item [-] FaceForensics++(FF++) \cite{rossler2019faceforensics++}: 与 FaceForensics(FF) 类似,该来源从 Youtube 平台取得 1000 个影像,同时使用 4 种方式进行伪造,而当中四种方法包含了 Deepfakes、Face2Face、FaceSwap、Neural Textures。

\item [-] Deepfake-TIMIT \cite{korshunov2018deepfakes}: 根据 Faceswap-GAN 方法进行伪造,同时该资料集也是第一个使用 GAN 所产生的伪造资料集。其资料是根据 VidTIMIT 来源去选 32 人,然后进行两两替换产生。

\item [-]Mesonet data \cite{afchar2018mesonet}: 从 Youtube 所产生的数据集。

\item [-]Celeb-DF \cite{li2019celeb}: 来源从 Youtube 进行搜集,同时考量 UADFV、FaceForensics++(FF++)、Deepfake-TIMIT等缺陷后,对伪造的方法进行改良。

\item [-] DeepfakeDetection(DFD) \cite{list1067}:由 Google 公司所制作的资料集,当中请 28 个演员来做出 363 个原始影像资料。

\item [-] DFDC preview Dataset \cite{dolhansky2019deepfake}: 由脸书在 The Deepfake Detection Challenge 所举办的比赛中所开放的测试资料集,当中有 5214 个影像。

\item [-] DFDC \cite{list1069}: 由脸书在 The Deepfake Detection Challenge 比赛所提供的正式资料集。

\item [-] DeeperForensics-1.0 \cite{jiang2020deeperforensics}:由南洋理工和商汤科技从 26 个国家收集 100 名演员的脸部数据,过程中将 FaceForensics++资料集中的 1000 笔原始影像作为目标影像进行训练,产生 50000 笔未修改影像跟 10000 笔修改影像。
\end{itemize}

\section{深度伪造的语音资料集}

\begin{itemize}
\item [-] ASVspoof 2015 database \cite{list1071}: 为因应语音伪造欺骗的问题,而在 2015 年所举办的 synthetic and converted speech 竞赛,当中开放第一个大规模的语音伪造资料集,该资料集根据 106 位不同的人的语音纪录,其分别为 45 名男性, 61 名女性组成的训练资料集有 3750 笔原始语音资料片段与 12625 组成的欺骗片段,另外验证资料集则有有 3497 笔原始语音资料片段与 49875 组成的欺骗片段,而最后测试资料集则为 9404 笔原始资料集与 184000 笔欺骗语音资料所组成。

\item [-] ASVspoof 2019 database \cite{list1072}: 2019 年所举办的 synthetic and converted speech 竞赛则根据 107 名不同的人士所撷取的原始资料,所建立起的资料集。
\end{itemize}

\begin{figure}[htb]
\centering 
\includegraphics[width=0.60\textwidth]{img/ch2m1.png} 
\caption{Li XR 等人 \cite{2021496} 的工作总结的资料对比}
\label{Test}
\end{figure}
