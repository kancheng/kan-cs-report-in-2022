\chapter{工作总结}
\label{chap:2}

本作业针对近年个人信息隐私保护等重要性日渐提升等原因,同时在面对近年来深度伪造等技术的发展带来对假讯息流通的挑战,对此针对该领域,在其检测与对抗性等工作进行梳理,同时在此说明近期的研究成果,其包含了多模态多尺度(Multi-modal Multi-scale) 和增量学习 (Incremental Learning) 所导入于检测领域的应用,另外也针对为了保护个人隐私所针对的语音检测对抗性跟应对语音攻击的深度伪造在语音检测与对抗性部分的研究进行整理,此外则是针对深度伪造在生成与检测的对抗性的部分针对两个不同的型态进行说明,最后则是因近年来深度伪造的成长,而讨论运用区块链对其视频进行追踪的手段的工作,将其作为在深度伪造与检测跟对抗性领域的安全措施进行说明。


%\begin{figure}[htb]
%\centering 
%\includegraphics[width=0.80\textwidth]{img/XXXXX.png} 
%\caption{XXXX}
%\label{Test}
%\end{figure}


%\begin{itemize}
%\item [-] L1
%\end{itemize}

%\begin{Verbatim}
%CODE
%\end{Verbatim}



%\begin{figure}[htb]
%\centering 
%\includegraphics[width=0.60\textwidth]{img/picname.png} 
%\caption{XXXX}
%\label{Test}
%\end{figure}
