\chapter{联邦学习与隐私安全}
\label{chap:2}

\section{联邦学习}

2016 年 Google 提出了 Federated Learning(联邦学习),能够让所有用户的手机「合作式」训练一个强大的模型,同时资料可以只要存在用户自己的手机中,维持了资料的隐私性,在这几年,随着隐私意识与法规的提升,Federated Learning也受到了各大公司、研究机构的重视。要搞懂Federated Learning有两个必备的知识,其一为资料隐私,其二为分布式机器学习(Distributed Machine Learning),接下来分别简单介绍这两者,并介绍Federated Learning的研究问题与第一个演算法FedAvg。


\section{资料隐私}

Google 在多年前推出一个服务,能够整合Google mail, map, calendar等应用的资讯,提供用户更自动化的服务,其意义在于,某人今天暑假想要到荷兰阿姆斯特丹远离都市生活看著名的风车村,从订好机票的那一个,信件中收到航空公司的确认信,同时 Google 用他的人工智慧系统从信件中也抓到了这个资讯,帮你把来回的机票时间标在 Google Calendar 上,并且把你下飞机后会搭的接驳车、想要居住的旅馆都标在 Google Maps 上,使用者不用再一个一个手动,Google一次帮你做好。虽然很方便,考虑到侵犯隐私时,可以快速想像另一个情景,今天 Google Map 同时也有使用者过去的用餐地点的资料,知道使用者喜欢吃什么、不喜欢吃什么,所以把使用者在阿姆斯特丹喜欢的餐厅全部都标出来推荐给你。而 Google 从使用者的搜寻纪录也知道使用者喜欢什么类型的景点,所以也帮使用者把在阿姆斯特丹的行程规划好。虽说 Google 贴心地安排了所有的行程,但也顺势地藉由 GPS 定位或是搜寻纪录,投放了周边商家的广告,在无形之中旅游行程的客制化与人们的隐私权之间的界线开始变得更模糊了。也许可以稍微反思,当人们的定位或搜寻纪录被利用在投放广告等其他用途的同时,是不是也代表着人们的隐私正在一点一滴的流失。

 另外客制化跟隐私的界线在于,客制化的前提是使用个人资料,问题是哪些资料可以被使用,哪些不可以被使用,被如何使用,应该要能由使用者决定。 2018 年纽约时报揭露,剑桥分析公司使用 5000 万个 Meta 上用户的资料进行建模,进而研究出每个用户的个性、偏好,进而应用在 Psychographic Targeting (心理变数行销),针对每个人最容易被煽动的角度客制化行销文宣,进而影响大选结果。同时 2019 年 Netflix 将这个事件改编成短影集「个资风暴:剑桥分析事件」,引起了大众对个人资料在网路上流通的重视度。而 2020 年哈佛大学社会心理学教授 Shoshana Zuboff 出版了「监控资本主义时代(The Age of Surveillance Capitalism: The Fight for a Human Future at the New Frontier of Power)」,讲解了在高度数位化、智能化的现在,企业如何使用个人资料来改变使用者的行为。

 
这一系列的事件其实都显示出了,公众意识到了随着科技快速发展,现有法规对个人资料的保护已经不够完整。所幸在 2016 年欧盟就率先制定了GDPR (General Data Protection Regulation) 来限制企业对个人资料 (Personal Data) 的使用,订立出了一个高标准要求企业,GDPR 的核心是「给用户自由选择资料是否被使用」,包含了「被遗忘权」 (Right to be forgotten)、「限制处理权」(Right to restriction of processing),而 GDPR 的出现也冲击了现今最火热的两大技术「深度学习」、「区块链」。随着越来越高的隐私意识,深度学习研究者们也相对应的开启了 Privacy ML 的研究热潮,而 Federated Learning 的核心就是为了保有隐私,让用户在「不用上传自己资料到企业端的同时可以训练模型、使用智能化服务」。

\section{分布式机器学习}

